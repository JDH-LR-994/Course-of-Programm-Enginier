\documentclass[a4paper]{article}
\usepackage[utf8]{inputenc} %Кодировка
\usepackage[T2A]{fontenc}
\usepackage[english, russian]{babel}
\usepackage{amsmath, amsfonts, amssymb} %Мат пакеты
\usepackage{fancybox, fancyhdr}
\pagestyle{fancy}
\fancyhf{}
\fancyfoot[R]{\thepage} %Справа снизу подпись номера страницы
\fancyhead[L]{Физика. Билет №27. Случайная величина. Среднее значение и дисперсия, функция распределения вероятностей.}
\setcounter{page}{1} %счётчик нумерации страниц
\headsep=10mm %Отступ 
\usepackage{xcolor}
\usepackage{hyperref}

\begin{document}
	
	\section*{Случайная величина}
	\begin{flushleft}
		Случайная величина - численный результат любого эксперемента или феномена (оценка за экзамен, снятие показателей приборов итд) \linebreak
		Она бывает: \linebreak
		1. Непрерывной (находиться на каком-то промежутке). Пример: температура в комнате \linebreak
		2. Дискретная (принимает какое-то значения из счётного или меньшего множеств) Пример: на игральном кубике выпало 3   
	\end{flushleft}
	\section*{Среднее значение и дисперсия}
	\begin{flushleft}
		Среднее значение: $\overline{x} = \lim_{N \rightarrow \infty} \frac{N_1 x_1 + ... N_n x_n}{N}=  \sum_{i=1}^{i=n}x_i \omega_i$, где $\omega_i = \frac{N_i}{N}$  \linebreak
		Дисперсия: $D = \overline{(x - \overline{x}) ^ 2} = \sigma ^ 2 = \overline{x^2} - \overline{x} ^ 2 $ \linebreak
		Дисперсия - это мера разброса данных от среднего значения или среднеквадратичное отклонение
	\end{flushleft}
	\section*{Функция распределения вероятности}
	\begin{flushleft}
		$x_{rand}$ - случайная величина \linebreak
		$\delta\omega(x_{rand} \in [x, x+ \delta x]) = \delta \omega (x_{rand})$ - вероятность, что случайная величина находится на отрезке
		\linebreak
		$\delta x \rightarrow dx, \qquad \delta \omega \rightarrow d \omega \approx dx$ \linebreak
		$d \omega = f(x) dx $, где: \linebreak
		 $f(x)$ - плотность вероятностей (функция, которая описывет вероятность, что случайная величина находится на интервале), \linebreak $d \omega $ - функция распределения (например, это может быть функция нормального или равномерных распределений)
	\end{flushleft}
	
\end{document}